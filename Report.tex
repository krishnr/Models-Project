%%%%%%%%%%%%%%%%%%%%%%%%%%%%%%%%%%%%%%%%%
% Journal Article
% LaTeX Template
% Version 1.4 (15/5/16)
%
% This template has been downloaded from:
% http://www.LaTeXTemplates.com
%
% Original author:
% Frits Wenneker (http://www.howtotex.com) with extensive modifications by
% Vel (vel@LaTeXTemplates.com)
%
% License:
% CC BY-NC-SA 3.0 (http://creativecommons.org/licenses/by-nc-sa/3.0/)
%
%%%%%%%%%%%%%%%%%%%%%%%%%%%%%%%%%%%%%%%%%

%----------------------------------------------------------------------------------------
%   PACKAGES AND OTHER DOCUMENT CONFIGURATIONS
%----------------------------------------------------------------------------------------

\documentclass[twoside,twocolumn]{article}

\usepackage{blindtext} % Package to generate dummy text throughout this template

\usepackage[sc]{mathpazo} % Use the Palatino font
\usepackage[T1]{fontenc} % Use 8-bit encoding that has 256 glyphs
\linespread{1.05} % Line spacing - Palatino needs more space between lines
\usepackage{microtype} % Slightly tweak font spacing for aesthetics

\usepackage[english]{babel} % Language hyphenation and typographical rules

\usepackage[hmarginratio=1:1,top=25mm,columnsep=25pt,textwidth=450pt]{geometry} % Document margins
\usepackage[hang, small,labelfont=bf,up,textfont=it,up]{caption} % Custom captions under/above floats in tables or figures
\usepackage{booktabs} % Horizontal rules in tables

\usepackage{lettrine} % The lettrine is the first enlarged letter at the beginning of the text

\usepackage{enumitem} % Customized lists
\setlist[itemize]{noitemsep} % Make itemize lists more compact

\usepackage{abstract} % Allows abstract customization
\renewcommand{\abstractnamefont}{\normalfont\bfseries} % Set the "Abstract" text to bold
\renewcommand{\abstracttextfont}{\normalfont\small\itshape} % Set the abstract itself to small italic text

\usepackage{titlesec} % Allows customization of titles
\renewcommand\thesection{\Roman{section}} % Roman numerals for the sections
\renewcommand\thesubsection{\roman{subsection}} % roman numerals for subsections
\titleformat{\section}[block]{\large\scshape\centering}{\thesection.}{1em}{} % Change the look of the section titles
\titleformat{\subsection}[block]{\large}{\thesubsection.}{1em}{} % Change the look of the section titles

\usepackage{fancyhdr} % Headers and footers
% \pagestyle{fancy} % All pages have headers and footers
\fancyhead{} % Blank out the default header
\fancyfoot{} % Blank out the default footer
\fancyfoot[RO,LE]{\thepage} % Custom footer text

\usepackage{titling} % Customizing the title section

\usepackage{hyperref} % For hyperlinks in the PDF

\usepackage{graphicx} % For inserting images
\graphicspath{ {images/} } % Folder where images are stored

\usepackage{amsmath}

%----------------------------------------------------------------------------------------
%   TITLE SECTION
%----------------------------------------------------------------------------------------

\setlength{\droptitle}{-4\baselineskip} % Move the title up

\pretitle{\begin{center}\Huge\bfseries} % Article title formatting
\posttitle{\end{center}} % Article title closing formatting
\title{SYDE 351 Project Report} % Article title
\author{%
\textsc{Krishn Ramesh, 20521942} \\[1ex]
\and
\textsc{Kevin Michael, 20507239} \\[1ex]
\and
\textsc{Peter Gokhshteyn, 20508507} \\[1ex]
}
\date{July 26, 2016} % Leave empty to omit a date

\renewcommand{\maketitlehookd}{%
\begin{abstract}
\noindent This report analyzed the behaviour of a dynamic system using bond graph modeling and MATLAB simulation. The system consists of two rotating wheels connected by an inextensible rope which exhibits slip due to a series of springs and a mass that connect the wheels. The modeling of the system was split into several stages to account for slip and the position of the mass was treated as the output of the system. Modeled results were very similar to actual results and it was observed that much of the movement was sinusoidal in nature. The final result is an imperfect oscillation that is a combination of two oscillations  and  the  relative  geometries  of  the  two wheels.
\end{abstract}
}

%----------------------------------------------------------------------------------------

\begin{document}

% Print the title
\maketitle

%----------------------------------------------------------------------------------------
%   ARTICLE CONTENTS
%----------------------------------------------------------------------------------------

\section{Introduction}

\lettrine[nindent=0em,lines=3]{I} nteresting dynamic behaviours can be found everywhere, but as simple as they may seem at first glance, their internal workings can be quite tricky to study. Systems Design Engineers, in particular, need to be proficient at modeling complex systems. To practice this skill, the SYDE 351 course has a modeling \& simulation project. A simple prototype was chosen to be built, taking inspiration from the chain mechanism of bicycles. The prototype\footnote{Video of the prototype in action can be seen here: \href{url}{youtu.be/hUUK9l1cf5w}}, seen in Figure 1, is composed of two wheels of varying sizes. Physically, the user rotates the top wheel which in turn rotates the bottom wheel. The bottom wheel slips suddenly every rotation due to the elastics that connect both wheels.

\begin{figure}[!ht]
    \caption{Photograph of the prototype}
    \centering
        \includegraphics[width=0.35\textwidth]{prototype.jpg}
\end{figure}

%------------------------------------------------

% Describe your prototype and any interesting things about how it was constructed, problems encountered, solutions you invented.
% Include a schematic diagram of the prototype
\section{Prototype}

The prototype features two rotating wheels of different diameters connected through two pathways. Firstly, they are connected by an inextensible rope which acts as a transformer transferring the angular velocity of the big wheel to the smaller. Secondly, both wheels have fixtures near their edges with are attached to elastics. The two elastics, which vary in length and strength, are connected together with a small mass in the middle. The prototype features mechanical rotation (turning wheels) and translation (moving mass). The wheels are made out of foam and the mass in the middle is a thin screw wrapped in duct tape. A schematic diagram of the prototype can be seen in Figure 2.

\begin{figure}[!h]
    \caption{Schematic diagram of the prototype}
    \centering
        \includegraphics[width=0.4\textwidth]{schematic.jpg}
\end{figure}

The user drives the prototype by rotating the handle of the larger wheel. Both wheels rotate smoothly initially, until the elastic gets stretched far enough to contract rapidly, causing the bottom wheel to slip. When the bottom wheel slips, the rope gains a lot of slack that would often get tangled around the axle. In order to address this problem, two guide rails were added (the two black pieces on the left side of Figure 1) consisting of an elevated tube that constrains the rope from accidentally slipping under the bottom wheel.


%------------------------------------------------

% Describe your BG, components, constitutive equations, and so on. Explain how you arrived at all the parameters you used. What  measurements did you take? How? Estimations and assumptions made? Theoretical calculations? Extraction of components and separate testing e.g. for springs?
\section{Model}

The bond graph of the prototype can be seen in Figure 3.
\begin{figure}[!h]
    \caption{Bond Graph of the prototype}
    \centering
        \includegraphics[width=0.5\textwidth]{bg.png}
\end{figure}

The user's input (i.e. turning the big wheel) is considered a source of flow as the user sets the angular velocity. The user will have to change the amount of torque they provide in order to account for the friction of the axle as well as the forces from the elastics. Because both wheels are connected via a taut inextensible rope (i.e. a transformer) and do not slip initially, they are in derivative causality as their angular velocity depends on the user's input velocity. However, when the bottom wheel slips and the rope gains slack, the rope is not taut and no longer acts as a transformer causing the bottom wheel to go into an integral causality state. The R element represents drag friction damping the rotation of the bottom wheel. Since the prototype requires very precise parameters to work properly, the parameters were chosen simply based on trial and error. The simulations assumes values for parameters, seen in Table 1 of the following section.

The modulated transformers represent the fixtures of the elastics on the wheels. As the fixtures move around the edge of the wheels, the distance between them changes, which results in a changing displacement of each elastic. The MTFs depend on $\theta_1$, according the following equations:

\begin{gather*}
x_1 = r_1 \sin(\theta_1) \\
y_1 = r_1 \cos(\theta_1) \\
x_2 = r_2 \sin(\theta_2) = r_2 \sin \left( \frac{r_1}{r_2} \theta_1 \right) \\
y_2 = r_2 \cos(\theta_2) - d = r_2 \cos \left( \frac{r_1}{r_2} \theta_1 \right) - d \\
\end{gather*}


%------------------------------------------------

% What are your state variables and outputs of interest. Give matlab results (plots, not charts of numbers). Do NOT provide massive quantities of outputs. Be very selective and apply engineering acumen. One good output plot may be sufficient. Providing stacks of output plots is not necessary, in fact undesirable and would receive a lower grade. State variables themselves are often not interesting outputs.
\section{Simulation}

The simulation was conducted in three separate stages. The first was the rotation of both wheels through the transfer function of the inextensible rope. The second stage of the simulation was the slip of the small wheel where the rope became loose and the wheel moved freely from the force of the elastics. The third stage continues through the rotation of the bigger wheel until the rope is wound tight again and the process can repeat.

\begin{figure}[!h]
    \caption{Simulation (top) vs. Testing (bottom)}
    \centering
        \includegraphics[width=0.5\textwidth]{x.png}
    \centering
        \includegraphics[width=0.5\textwidth]{y.png}
    \centering
        \includegraphics[width=0.45\textwidth]{tracking-plots.png}
\end{figure}

Figure 4 shows the x and y positions of the oscillating mass respectively over time. The three stages are shown in the graphs using red lines as dividers between them. As can be seen, the motion of the mass is first smooth, moving in a sinusoidal fashion until the slip occurs and there begins a damped oscillation. Once the oscillation stops, the big wheel continues to spin moving the mass in a sinusoidal fashion once more.

The simulation was modeled using 4 state variables - the displacement of the two springs ($x_3, x_4$), the momentum of the mass between them ($p_5$) and the moment of inertia of the bottom wheel ($p_2$). The values chosen for each of the variables in the system are shown in Table 1.

\begin{table}
\caption{Variable values chosen}
\centering
\begin{tabular}{lr}
\toprule
\cmidrule(r){1-2}
Variable & Value \\
\midrule
Top Spring Constant & $20$ $N/m$ \\
Bottom Spring Constant & $30$ $N/m$ \\
Mass Between Springs & $0.5$ $kg$ \\
Small Wheel Moment of Inertia & $25$ $kg \cdot m^2$ \\
\bottomrule
\end{tabular}
\end{table}

The system was simple to model for the first and last stage as all the springs and masses were in static equilibrium at every moment, meaning that the two wheels spun at constant angular velocities and the mass would move according to the applicable geometry. During the slip, the movement of the bottom wheel due to the applied moment resulted in the bottom spring contracting, creating an imbalance in equilibrium for the mass. This in turn moves the mass and causes it to oscillate, while also causing the wheel to oscillate as well. Therefore, the final result is an imperfect oscillation that is actually a combination of two oscillations and the relative geometries of the two wheels.

%------------------------------------------------

% Compare simulation to data obtained from the system. Describe how well your simulation matches the actual behaviour as determined by an easily observable representative output variable (flashing lights, buzzer, counting rotations, and so on). Could the results be improved? Speculate on any significant descrepancies between test observations and simulation outputs. Explaining your validation process and demonstrating your understanding of the methods are much more important than getting wonderfully close agreement with the behaviour of your prototype.
\section{Discussion}

By comparing the two graphs created from the simulation to the ones created from tracking the motion of the mass, it can be seen that they are very similar. Firstly, both stages 1 and 3 have similar sinusoidal movements, though the ones from the real model are less smooth. This is mostly because of the inconsistency in the source of flow that is the hand moving the top wheel. Moreover, it can be seen that the amplitudes and relative phases of the sinusoidal waves are different. This is because of the inaccurate measurement of the geometries of the wheels. Although measuring the radii of the wheels is trivial, they are not constant as the winding of the rope changes the dimensions of the wheels.

Looking at stage 2 the graphs are also very similar, though the oscillation in the simulation is again much smoother and even than the one in the real model. This is mostly because the elastics are not perfectly linear springs and thus do not act like they are modeled in the simulation. However, the oscillations die down at approximately the same rate, meaning the damping was chosen fairly accurately.

\end{document}